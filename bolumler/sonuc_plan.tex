\chapter{Sonuç ve Gelecek Çalışma Planı}
\label{ch:sonuc}

Bu tez çalışmasında (Bitirme Projesi I), GPS sinyallerinin erişilemez olduğu veya güvenilmez hale geldiği durumlarda, çoklu otonom araç sürülerinin işbirlikli olarak konumlarını tespit edebilmesi için gerekli teorik altyapı oluşturulmuştur.

\section{Elde Edilen Kazanımlar}
Yapılan literatür taraması, sistem modellemesi ve yöntem analizleri sonucunda şu çıktılara ulaşılmıştır:
\begin{enumerate}
    \item Mimari Kararı: Merkezi bir başarısızlık noktasından kaçınmak ve sistemin beka kabiliyetini artırmak (savunma sanayi gereksinimleri) amacıyla Dağıtık (Decentralized) mimari seçilmiştir.
    \item Algoritma Seçimi: Araçların ani manevra yapabileceği senaryolarda, lineerleştirme hatalarını minimize etmek için EKF yerine türevsiz Sezgisel Kalman Filtresi (UKF) yöntemi benimsenmiştir.
    \item Veri Tutarlılığı: Dağıtık sistemlerdeki "veri tekrarı" (data incest) problemini çözmek için sisteme Kovaryans Kesişimi (Covariance Intersection) algoritması entegre edilmiştir.
\end{enumerate}

\section{Gelecek Çalışma Planı (Bitirme Projesi II)}
Projenin ikinci aşamasında, bu dönem kurgulanan matematiksel modelin gerçek zamanlı simülasyon ortamında doğrulanması hedeflenmektedir. Geliştirme süreci, karmaşıklığın kademeli olarak artırıldığı (incremental) bir yol haritası izleyecektir.

\subsection{Simülasyon Ortamı ve Araçlar}
\begin{itemize}
    \item Yazılım Altyapısı: Simülasyon, robotik uygulamalarda endüstri standardı olan ROS 2 (Humble/Foxy) üzerinde geliştirilecektir.
    \item Fizik Motoru: Araçların dinamiğini ve sensör verilerini simüle etmek için Gazebo ortamı kullanılacaktır.
    \item Robot Platformu: Hazır sensör paketlerine (Lidar, IMU, Odometre) sahip olması nedeniyle TurtleBot3 modeli tercih edilmiştir.
\end{itemize}

\subsection{Aşamalı Geliştirme Stratejisi}
Yazılım geliştirme sürecinde hataların izole edilebilmesi için şu sıralama izlenecektir:

\begin{enumerate}
    \item \textbf{Aşama 1: İdeal Ortamda Doğrulama (Gürültüsüz):}
    İlk aşamada simülasyon gürültüsü kapatılarak, tekil aracın hareket denklemlerinin ve UKF tahmin (prediction) bloğunun kodlaması doğrulanacaktır. Amaç, simülasyon verisi ile algoritma çıktısının birebir örtüştüğünü görmektir.
    
    \item \textbf{Aşama 2: Gürültü Altında Tekil Filtreleme:}
    Sensörlere yapay Gauss gürültüsü (Gaussian Noise) eklenecektir. UKF algoritmasının bu gürültüyü ne kadar filtreleyebildiği, "Ground Truth" (Gerçek Konum) verisi ile kıyaslanarak test edilecektir.
    
    \item \textbf{Aşama 3: İkili Araç ve İletişim Testi:}
    Simülasyona ikinci bir araç eklenecektir. Araçların V2V haberleşme üzerinden birbirlerine gönderdikleri konum paketlerinin doğruluğu ve göreceli ölçüm (Mesafe/Açı) modelleri gürültüsüz ortamda test edilecektir.
    
    \item \textbf{Aşama 4: Tam Entegrasyon ve Stres Testi:}
    Çoklu araç (3+) senaryosunda tüm gürültüler ve iletişim gecikmeleri aktif edilecektir. Dağıtık veri füzyonu ve Kovaryans Kesişimi algoritmalarının başarısı, sistemin toplam RMSE (Hata Kareler Ortalaması) değeri üzerinden raporlanacaktır.
\end{enumerate}

Bu strateji sayesinde, olası yazılım veya matematiksel hataların kaynağı (algoritma, sensör, iletişim) kolaylıkla tespit edilebilecektir.