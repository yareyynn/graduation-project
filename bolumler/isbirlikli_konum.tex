\chapter{Matematiksel Model ve Sistem Mimarisi}
\label{ch:sistem_modeli}

Bu bölümde, işbirlikli konumlandırma probleminin matematiksel altyapısı, araçların hareket kinematiği, sensör ölçüm modelleri ve önerilen dağıtık iletişim protokolü detaylandırılmıştır.

\section{Problem Tanımı ve Varsayımlar}
Tez kapsamında, GPS erişiminin olmadığı (GPS-denied) veya güvenilmez olduğu bir ortamda hareket eden $N$ adet otonom araçtan oluşan bir sürü (swarm) ele alınmıştır.
Sistemin tutarlılığı ve uygulanabilirliği için aşağıdaki varsayımlar kabul edilmiştir:

\begin{enumerate}
    \item \textbf{Tanımlanabilirlik:} Her araç, kendisine atanan benzersiz bir kimlik (ID) numarasına sahiptir ve iletişim paketlerinde bu kimliği yayınlar.
    \item \textbf{Eş Zamanlılık:} Araçların saatleri senkronizedir veya iletişimdeki gecikmeler zaman damgası (timestamp) ile takip edilebilir.
    \item \textbf{Tam Bağlı Ağ (All-to-All):} İletişim menzili içerisindeki her araç, yayınlanan paketleri alabilir.
    \item \textbf{Sabit Koordinat Çerçevesi:} Tüm araçlar göreve başlarken ortak bir yerel referans çerçevesine (Local Frame) göre hizalanmıştır.
\end{enumerate}

\section{Araç Kinematik Modeli}
Sürüdeki her bir araç ($i$), 2 boyutlu düzlemde hareket eden diferansiyel sürüşlü (veya Ackermann manevra kabiliyetine sahip) bir robot olarak modellenmiştir. Aracın $k$ anındaki durum vektörü $x_{k}^{(i)}$, konumu ve yönelimini içerir:

\begin{equation}
    x_{k}^{(i)} = \begin{bmatrix} p_x \\ p_y \\ \theta \end{bmatrix}_k^{(i)}
\end{equation}

Aracın hareket modeli, doğrusal olmayan (non-linear) bir süreçtir. Diskret zamanlı hareket denklemleri şu şekildedir:

\begin{equation}
    x_{k+1}^{(i)} = f(x_{k}^{(i)}, u_{k}^{(i)}) + w_{k}^{(i)}
\end{equation}

Burada $u_{k}^{(i)} = [v, \omega]^T$ kontrol girdisini (doğrusal hız ve açısal hız), $w_{k}^{(i)}$ ise Gauss dağılımına sahip süreç gürültüsünü temsil eder. Açık formda yazılırsa:

\begin{equation}
    \begin{bmatrix} p_x \\ p_y \\ \theta \end{bmatrix}_{k+1} = 
    \begin{bmatrix} p_x \\ p_y \\ \theta \end{bmatrix}_{k} + 
    \begin{bmatrix} v \Delta t \cos(\theta) \\ v \Delta t \sin(\theta) \\ \omega \Delta t \end{bmatrix}_{k}
\end{equation}
Bu model, bir sonraki bölümde anlatılacak olan Sezgisel Kalman Filtresi'nin (UKF) "Tahmin" (Prediction) aşamasında kullanılacaktır.

\section{Göreceli Ölçüm Modeli}
Her araç, üzerindeki LiDAR veya Radar sensörleri aracılığıyla çevresindeki diğer araçları tespit edebilir. Araç $i$'nin Araç $j$'yi algıladığı durumdaki ölçüm vektörü $z_{ij}$, aradaki mesafeyi ($r_{ij}$) ve açıyı ($\phi_{ij}$) içerir:

\begin{equation}
    z_{ij} = h(x_i, x_j) + \eta_{ij} = 
    \begin{bmatrix} 
    \sqrt{(p_x^j - p_x^i)^2 + (p_y^j - p_y^i)^2} \\ 
    \text{atan2}(p_y^j - p_y^i, p_x^j - p_x^i) - \theta^i 
    \end{bmatrix} + \eta_{ij}
\end{equation}

Burada $\eta_{ij}$, ölçüm sensöründeki gürültüyü (Measurement Noise) ifade eder. Bu model, filtrenin "Düzeltme" (Correction) aşamasında kullanılacaktır.

\section{Dağıtık İletişim ve Veri Paylaşım Protokolü}
Merkezi bir otoritenin bulunmadığı ve tekil başarısızlık noktalarının (Single Point of Failure) engellendiği bu sistemde, "Durum Paylaşımı" (State Exchange) yöntemi benimsenmiştir.

Araçlar, ham sensör verilerini (raw data) paylaşmak yerine, kendi işlemcilerinde hesapladıkları durum kestirimlerini paylaşırlar. Bu yaklaşım, iletişim bant genişliğini (Bandwidth) verimli kullanmak ve ağ yoğunluğunu önlemek için kritiktir. 

Bir araç ($i$) tarafından yayınlanan ve diğer araçlarca dinlenen veri paketinin yapısı şöyledir:

\begin{table}[h!]
\centering
\caption{V2V İletişim Paketi İçeriği}
\label{tab:veri_paketi}
\begin{tabular}{|l|l|l|}
\hline
\textbf{Veri Alanı} & \textbf{Sembol} & \textbf{Açıklama} \\ \hline
Araç Kimliği & $ID_i$ & Gönderen aracın benzersiz numarası. \\ \hline
Durum Kestirimi & $\hat{x}_i$ & Aracın kendi hesapladığı Konum ve Yönelim $[x, y, \theta]^T$. \\ \hline
Kovaryans Matrisi & $P_i$ & Aracın kendi konumuna ne kadar güvendiğini gösteren hata matrisi. \\ \hline
Zaman Damgası & $t_k$ & Verinin oluşturulduğu an. \\ \hline
\end{tabular}
\end{table}

Bu protokol sayesinde, alıcı araç ($j$); hem gönderici aracın ($i$) nerede olduğunu iddia ettiğini ($\hat{x}_i$) öğrenir, hem de bu bilgiye ne kadar güvenmesi gerektiğini ($P_i$) analiz edebilir.